\documentclass{article}
\renewcommand{\baselinestretch}{1}
\setlength{\textheight}{9in}
\setlength{\textwidth}{6.0in}
\setlength{\headheight}{0in}
\setlength{\headsep}{0in}
\setlength{\topmargin}{0in}
\setlength{\oddsidemargin}{0in}
\setlength{\evensidemargin}{0in}
\setlength{\parindent}{0.3in}
\pagestyle{empty}
\begin{document}

\leftline{\bf Technical Review} 

\leftline{Andrew Oliva}

\leftline{AME 60611}

\leftline{25 September 2015}

\bigskip
It is often desired to take a complicated system of equations and reduce them,
 by some means, to a more easily modelled and easily studied set of equations.  
 One such technique is to \textit{regularize}\footnote{Neumaier, A., 1998, Solving
 Ill-conditioned and Singular Linear Systems: A Tutorial on Regularizations, {\em SIAM Review} 40: 636-666.}
 the set equations and study the
 output.  Which is relevant, so long as they still obey the appropriate physical
 laws that the original set of equations was meant to embody.
\\

Perhaps one of the most notoriously difficult set of equations to regularize is
the Navier-Stokes equations, which govern the conservation of mass, momentum,
and energy in a general fluid (i.e. compressible, viscous, unsteady, turbulent, \textit{etc}.)
 As a step toward regularizing the Navier-Stokes equations, Guermond, {\em et al.},\footnote{Guermond, J., Popov, B., 2014,
     Viscous Regularization of the Euler Equations and Entropy Principles,
 {\em SIAM Journal on Applied Mathematics}, 74(2): 284-305.}
 have shown regularized Euler equations (which assume inviscid flow; $\mu~=~0$).
 In addition, and perhaps most importantly, their work proves that with such a 
 regularization, the density and internal energy are positive definite while
 simultaneously proving that specific entropy satisfies a minimum principle. 
 Meaning that the regularization obeys the same physical laws that the original
 Euler equations were meant to capture. The Euler equations in their regularized form
 are given below as:

\begin{equation}
    \partial_t \rho + \nabla \cdot (a \nabla \rho) = 0
\end{equation}
\begin{equation}
    \partial_t \mathbf{m} + \nabla \cdot(\mathbf{u} \otimes \mathbf{m}) + pI) - \nabla\cdot(a\nabla\rho \otimes
    \mathbf{u} + G(\nabla^s\mathbf{u})) = 0
\end{equation}
\begin{equation}
    \partial_t E + \nabla \cdot (\mathbf{u}(E + p)) - \nabla \cdot (a\nabla(\rho e) + \frac{1}{2} \mathbf{u}^2 a \nabla p + G(\nabla^s\mathbf{u}) \cdot \mathbf{u} = 0
\end{equation}

Equations (1), (2), and (3) represent the conservations of mass, momentum, and energy, respectively.
Where $\partial_t$ is the partial derivative with respective to time, $\rho$ is density, $\mathbf{u}$ is the velocity,
$G$ is a flux term  (which results from the regularization and has different forms for different assumptions), $E$ is the internal energy, $p$ is pressure, $I$ is the identity matrix, and $a$ is a function that satisfies the specific
 entropy minimum principle (not shown here).
\\

 The form shown above for these equations is, according to the authors, the most robust form of the regularization of the Euler equations. Thus, the equations above are more easily
 implemented in a numerical scheme for evaluation and study. This form of the Euler equations is then explored over a range of values and also compared to the classical Navier-Stokes regularized equations (which are not nearly as easily solved as the regularized Euler equations presented here).
\\

 Ultimately, these regularizations push the field of fluid mechanics toward regularizing the Navier-Stokes equations. However, the Navier-Stokes equations have an added
 complication that the entropy of the fluid is not universally guaranteed to increase with a given process. The thermal flux term can allow the fluid to actually decrease in entropy.
  This will not violate the second law of thermodynamics so long as the surrounding system increases in entropy.
  It is this removal of the minimum principle of the specific entropy term that complicates the regularization of the Navier-Stokes equations.
  However, the work in this paper makes significant strides in identifying specific terms that can simplify the regularization. It also suggests that
  with further work, additional improvements can be made in our understanding of how to apply the technique appropriately.
  Which was demonstrated by utilizing this regularization technique to the Euler equations.
\\

If these techniques were to be fully realized an easily implemented numerical scheme could be developed to rapidly solve the Navier-Stokes equations which would result in orders of
magnitude speed up while still maintaining sufficient accuracy.


\end{document}
