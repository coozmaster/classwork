\documentclass{article}
\usepackage{epsfig}
\renewcommand{\baselinestretch}{1}
\setlength{\textheight}{9in}
\setlength{\textwidth}{6.5in}
\setlength{\headheight}{0in}
\setlength{\headsep}{0in}
\setlength{\topmargin}{0in}
\setlength{\oddsidemargin}{0in}
\setlength{\evensidemargin}{0in}
\setlength{\parindent}{.3in}
\begin{document}

\leftline{Andrew Oliva}
\leftline{AME 60611}
\leftline{28 August 2015}

\bigskip
This is a sample file for the text formatter \LaTeX.
I require you to use \LaTeX~for the following reasons:

\begin{itemize}

\item
It produces the best output of text, figures,
and equations of any
program I have seen.

\item
It is machine-independent.
You can e-mail ASCII versions of most relevant files.

\item
It is the tool of choice for many research
scientists and engineers.
Many journals accept 
\LaTeX~submissions, and many books
are written in \LaTeX.

\end{itemize}
\medskip
Some basic instructions are given below.
Put your text in here.  You can be a little sloppy    about
spacing.  It adjusts the text to look good.
{\small You can make the text smaller.}
{\tiny You can make the text tiny.}
Skip a line for a new paragraph.   

You can use italics ({\em e.g.} {\em  Math is everywhere}) or {\bf bold}.
Greek letters are a snap: $\Psi$, $\psi$,
$\Phi$, $\phi$.  Equations within text are easy---
The equation of a straight line is $y = m x+b$.
You can also set aside equations like so:
\begin{eqnarray}
\nabla \cdot \bf u &=& 0, \\
{dT_n \over dt} &=& \sum_{n=1}^N \left(-\mu - n^2\pi^2\right)T_n(t).
\end{eqnarray}
References\footnote{Lamport, L., 1986, {\em \LaTeX: User's Guide \& Reference Manual},
    Addison-Wesley: Reading, Massachusetts.}
are available. 
If you have a postscript file, say {\tt sample.eps}, in the same local directory,
you can insert the file as a figure.  Figure \ref{sample}, below, gives plots of various Bessel functions. 
\begin{figure}[h]
\centerline{\psfig{figure=sample.eps,height=1.2in}}
\caption{Sample figure plotting Bessel functions}
\label{sample}
\end{figure}

\medskip
\leftline{\em Running \LaTeX}
\medskip

You can create a \LaTeX~ file with any text editor ({\tt vi, emacs, gedit},
etc.). 
To produce a document, you need to run the \LaTeX~application
on the text file.  The text file must have the suffix ``{\tt .tex}''
On the Linux system this is done via the command

\medskip
{\tt latex2pdf file.tex}

\medskip
\noindent
This generates the file {\tt file.pdf}.

\bigskip
\noindent
Alternatively you can use {\tt TeXShop} on a Macintosh or {\tt MiKTeX} on a Windows-based machine.
The {\tt .tex} file must have a closing statement as
below.

\end{document}
